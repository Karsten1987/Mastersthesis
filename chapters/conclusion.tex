\chapter{Conclusion}
\label{chap:conclusion}

In this Master's thesis, we successfully implemented a complete solution for collision avoidance inside a hierarchical quadratic program. We integrated the Stack-of-Tasks efficiently inside a ROS-based environment and were therefore able to run it in real time on humanoid robots, such as REEM-H and REEM-C. \\

We implemented a reactive collision avoidance as an inequality constraint inside the quadratic program. Hereby, the velocity along the directional vector between the closest point pair of two collision objects is damped until a specified safety distance is reached. In order to avoid any singularities in calculating this closest point pair, we carried out a complete capsule decomposition for the robots collision model. The pseudo convexity of capsules lead to a smooth and moreover continuous trajectory of point pairs. We could prove, by various experiments, that the proposed solution is sufficiently stable to avoid any self-collision as well as deliver support for external collision objects. Thus, we could with great success accomplish all three goals of this work (see chapter \ref{item:goals}).

Equally, we freely admit, that the proposed method cannot be migrated as a out-of-the-box solution to different robots. Reasons for this is the strong dependency on the robot description file, which has to provide the support for capsules as a collision object. Furthermore, as we could see during the experiments, special care has to be taken for the control gain of the velocity damping. Those two configuration steps have to be calibrated for each individual robot.

The presented solution assures a safe execution of motion at all time. It ensures that all possible self-collisions are successfully avoided and the minimal distance threshold does not get violated (except numerical errors, which have to be taken into considerations). The collision avoidance thus is placed on a high level inside the hierarchy as a self-collision should never be violated. Since the velocity damping is realized as an inequality constraint, its result denotes a solution space. The suggested method provides a solution space, which is sufficiently large to not restrict the actual manipulation goals as long as they are outside the critical safety zone. It can thus be placed transparently on the top of the stack. It prevents all self-collision, but it flexible enough to not interfere the execution of beneath placed tasks.

The implementation for the humanoid robot from PAL Robotics REEM-H comprises a basic set of activated tasks on highest level, which are strictly necessary to allow a safe usage. On highest priority, joints limits have to be in range at all time in order not to physically harm the hardware of the robot. In particular, joint limits are meant as joint position as well as joint velocity limits. This is directly followed by the self-collision avoidance tasks. With this basic stack, a safe application can be performed on lower prioritized levels.

In general, the Stack-of-Tasks achieves very convincing performance in terms of whole body motion control. The execution speed of the inverse kinematic solver is by far ahead of any current implementations. This allows a smooth and application-driven usage of humanoid robots in a whole body control. The dynamic graph enables an easy \textit{plug-and-go} of input and output signals, which allows a straight-forward instantiation of fairly complex applications. \\
However, limitations arise on practical application aspects. Since the HCOD is a numerical solver, no look-ahead planning is done. This means, that even goal positions, which are clearly out of reach are solved to the best, minimizing the error between actual and goal position. This might lead to inappropriate behavior. Similarly, the DG operates in a constant control loop. Thus, emergency handling such as an immediate stop of the controller has to be handled with respect of this loop. 
